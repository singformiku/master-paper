\begin{abstracten}
\setcounter{page}{1}
Algriculture plays a key role in human history. Increasing of food productivity allows people to develop technology and civilization. Although been benefited from modern technology, such as agricultural machinery and farming methods, genetic technology, techniques for achieving economies of scale in production. We still face some problem, for example, farm security, population increasing, plant diseases, cattle diseases and abnormal behaviour. Increasing productivity and decreasing diseases damage of food resource in crop and stock is crucial mission for us. Previous research have used AI technology to do image recognition and classification for crop related research by collecting photo as training data. But if it comes to stock farming, this might become issue. Although some analysis about animal can be done by photo, abnormal behaviour is hard to analyze because continous behaviour is the range of actions and mannerisms. In other words, it is hard to be recognized by a single photo. Video training data is kind of thing we need. In this thesis, we proposed a recording streaming system which is efficient, automatic and cloud based. It can not only record manually but also preset time scheduling for specific timing to record and register events to trigger record process when sonething critical happened. Also, we use AWS as our cloud service which improve storage limit, difficulty of maintaining system and developing new feature etc.. At last, We can manage all of the camera in experimental field located all over Taiwan.
\end{abstracten}

\begin{abstractzh}
\setcounter{page}{2}
農業在人類文明發展上扮演著極為重要的角色,生產力的提升讓人類有餘力去發展創新科技。但現代中,雖有近代科技輔助,像是機械化農機具,基因改造科技和殺蟲劑等幫助。人們仍然面對諸如農場安全、人口劇增、作物疾病、牲畜疾病和動物異常行為等問題。因此增加食物產量及降低疾病帶來的損失是目前在農業和畜牧業的一個重要任務。前人的研究中使用照片訓練集來訓練AI影像辨識來達成對作物的相關分析研究。在畜牧業情況就變得比較複雜,雖然有一些動物相關的AI分析,例如動物品種辨識,也可以用照片來做訓練,但如果是跟動物行為就比較難用照片當作訓練集。因為動物的行為是一連串連續的動作,難以用單一照片來判斷行為的區別。本論文中設計並實作出一套基於雲端服務的全自動錄影串流系統,來解決收集影像資料的問題。該系統可以預約錄影時間,
讓攝影機在特定時間收集資料,也可以註冊事件觸發,讓攝影機能在特定重要事件觸發時開啟錄影功能記錄當下狀況,並也可以手動開啟錄影功能。最後我們使用AWS雲端整合服務,解決儲存空間限制、降低維護系統及開發新功能的難度,並可以同時管理全台灣多個場域的攝影機。

% 在此平台上,可以將農地的各種資料(如:土壤酸鹼值、濕度、農作物照片等....)藉由對應的感應器和網路監控攝影機上傳至平台上,並製作成圖表供專家做系統化分析; 或將之收集成資料集,用以訓練AI視覺影像的模型。以此提升農作物生產量。但平台仍然有許多可以改善的地方,如監控攝影機的操作穩定性、邊緣裝置的故障通知及邊緣裝置的可擴充難度等。且如果有研究或分析需要用到影片相關資料,以現有的平台無法提供相關的解決方案。此論文中將以各個角度去改善系統中現有的問題,經實驗驗證,可以讓網路監控攝影機的指令掉包率從23\%降到0\%,並實作裝置故障通知以降低流失資料帶來的損害,最後改善邊緣裝置的可擴充性。並且研發出一套錄影系統,以完成錄影資料的相關收集。
\end{abstractzh}



% \keywords{Optical Music Recognition, Pattern Recognition, Music Technology}

% \begin{comment}
% \category{I2.10}{Computing Methodologies}{Artificial Intelligence --
% Vision and Scene Understanding} \category{H5.3}{Information
% Systems}{Information Interfaces and Presentation (HCI) -- Web-based
% Interaction.}

% \terms{Design, Human factors, Performance.}

% \keywords{Region of interest, Visual attention model, Web-based
% games, Benchmarks.}
% \end{comment}
